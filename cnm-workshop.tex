\documentclass[12pt]{article}
\usepackage[margin=1.0in]{geometry}

\title{Superconducting Detectors and Electronics Workshop at CNM}
\author{W. Armstrong, A. Camsonne, K. Hafidi}
\begin{document}

\maketitle

Since superconductivity is based on loosely bound Cooper pairs of electrons, 
superconducting detectors and electronics are much more sensitive and 
responsive than typical semiconductor detectors. The applications for these 
developing technologies are very broad and have a high potential for 
innovation and novel uses. Some examples include superconducting Transition Edge 
Sensors (TES) that have energy resolutions of a few eV, and the faster 
Superconducting Tunnel Junction (STJ) having a resolution around 10 eV. 
Magnetic field detection is also achieved using Josephson Junctions in Single 
Quantum Interferometer Devices. The superconducting Single Nanowire Strips 
technology produces single photon detectors with efficiencies above 90 \% at 
photon wavelengths of 1550 nm with excellent timing resolution.  Such detectors 
can also be used for time of flight mass spectrometry of large molecules.  Many 
of these new technologies have direct applications for X-ray measurements at 
the APS and nanophotonics measurements at the CNM.

We are proposing a workshop on superconducting detectors and electronics.
Such a workshop will be of interest to users who are looking for very high 
energy or timing resolution measurements. A lot of R\&D collaboration 
opportunities are available including improving detection efficiency and timing, 
readout electronics, and investigation of operational performance under 
various conditions like high radiation and magnetic field environments.
Development of manufacturing techniques, in particular, towards high density 
detector arrays is an important area for collaboration among users and would 
be very beneficial as the technology matures. Typical detectors are of 
the scale of 100s of nanometers to micrometers requiring a lot of elements to cover 
a larger surface. Therefore integrated superconducting electronics are a natural 
extension to superconducting detectors to take full advantage of the speed, 
radiation hardness, and to simplify the interfacing of the detectors to the outside 
world by processing the informations as much as possible close to the 
detector. Superconducting electronics technologies will also be presented along 
with the progress on detector R\&D efforts.

\vspace{2cm}

\section{Proposed invited speakers}
\begin{itemize}
\item Francesko Marsili , NASA JPL
\item Joel Ullom	, NIST
\item Sae Woo Nam	, NIST
\item Clarence Chang 	, Argonne
\item Eric Dauler	, Lincoln Lab
\item Michael Siegel	, Karlsruhe
\item Peter  Day	, NASA JPL
\item Samuel Moseley	, NASA Goddard
\item Andrey Elagin	, U. Chicago
\item Deepnaryan Gupta  , HYPRES
\item John Musson       , JLab
\item Alec Sandy        , Argonne
\end{itemize}

\end{document}
