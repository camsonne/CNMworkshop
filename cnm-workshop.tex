\documentclass{article}
\title{Superconducting Detectors and Electronics Workshop at CNM}
\author{W. Armstrong, A. Camsonne, K. Hafidi}
\begin{document}

\maketitle

Since superconductivity is based on loosely bound Cooper pairs of electrons, 
superconducting detectors and electronics are much more sensitive and 
responsive than typical semiconductors detectors. The applications for these 
developing technologies are very broad and have a high potential for 
innovation and novel uses. Some examples include superconducting Transition Edge 
Sensors (TES) that have energy resolutions of a few eV, and the faster 
Superconducting Tunnel Junction (STJ) having a resolution around 10 eV. 
Magnetic field detection is also achieved using Josephson Junctions in Single 
Quantum Interferometer Devices. The superconducting Single Nanowire Strips 
technology produces single photon detectors with efficiencies above 90 \% at 
photon wavelengths of 1550 nm with excellent timing resolution.  Such detectors 
can also be used for time of flight mass spectrometry of large molecules.  Many 
of these new technologies have direct applications for X-ray measurements at 
the APS and nanophotonics measurements at the CNM.

We are proposing a workshop on superconducting detectors and electronics.
Such a workshop will be of interest to users who are looking for very high 
energy resolution measurements or very high timing resolution. A lot of R\&D 
opportunities are available such as improving detection efficiency and timing, 
readout electronics, and investigation of operational performance in under 
various conditions like high radiation and magnetic field environments.
There is also R\&D work to develop manufacturing techniques in particularly to 
go toward high density array of detectors. Involvement of users in this R\&D 
would be very beneficial to this field. In addition typical detectors are 
of the scale of 100 of nanometers to micrometers requiring a lot elements to 
cover a larger surface so integrated superconducting electronics is a natural extension to superconducting detectors to take fully advantage of the speed, radiation hardness and simplify the interfacing of the detectors to the outside world by processing the informations as much as possible close from the detector. Superconducting electronics techniques will be presented along with the R\&D topics relevant to make this kind of electronics available for mainstream use.
\end{document}
