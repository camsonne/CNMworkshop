\documentclass{article}
\title{Superconducting detectors and electronics workshop at CNM}
\author{W. Armstrong, A. Camsonne, K. Hafidi}
\begin{document}
\maketitle
Since superconductivity is based on loosely bound Cooper pairs of electrons, superconducting detectors can be much more sensitive than semiconductors detectors : Transistion Edge Sensors have resolutions of a few eV and Superconducting Tunnel Junction are faster with resolution around 10 eV. Detection of magnetic fields were also achieve using Josephson Junctions in Single Quantum Interferometer Devices. The Single Nanowire Strips technology produces single photon detector with efficiency above 90 \% at 1550 nm with excellent timing resolution. Such detectors can also be used for time of fligh mass spectrometry of large molecules.
Many of those detectors have direct applications for the measurements at the APS for X-ray measurements and at CNM for nanophotonics measurements for example.
We are proposing a workshop on superconducting detectors and electronics.
Such a workshop will be interesting for users who are looking for very high energy resolution measurement, or very high resolution timing which might consider the use of such detectors.
Since a lot of R\&D possible to improve the detector properties such as efficiency, timing resolution, operating temperature and value of critical current. There is also R\&D work to develop manufacturing techniques in particularly to go toward high density array of detectors. Involvment of users in this R\&D would be  very beneficial to this field. In addition to  typical detectors are of the scale of 100 of nanometers to micrometers requiring a lot elements to cover a larger surface.
\end{document}
